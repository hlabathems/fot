%\documentclass[12pt,preprint]{aastex}
\documentclass{emulateapj}
%\usepackage{apjfonts}
\usepackage{natbib}
\usepackage{graphicx}
\usepackage{bm}

\slugcomment{}

%% macros
\newcommand\da{\delta\!\alpha}
\newcommand\ks{\kappa_s}
\newcommand\ps{\phi_{\rm sub}}
\newcommand\mathbi[1]{\textbf{\em #1}}
\newcommand\rv{\bm r}
\newcommand\xv{\bm x}
\newcommand\uv{\bm u}
\newcommand\du{\delta\uv}
\newcommand\av{\bm \alpha}
\newcommand\dphi{\delta\phi}
\newcommand\dtau{\delta\tau}
\newcommand\avg[1]{\left\langle{#1}\right\rangle}
\newcommand\Rein{R_{\rm Ein}}
\newcommand\Sigcr{\Sigma_{\rm crit}}
\newcommand\tot{{\rm tot}}
\newcommand\mhat{{\hat m}}

\shorttitle{Inferring Time Delays in Strong Lenses}
\shortauthors{L.A.~Moustakas \& A.~Romero-Wolf}

\begin{document}

\title{Bayesian Inference Time Delay Recovery for Strong Gravitational Lenses}

\author{LAM \& ARW\altaffilmark{1}}
\altaffiltext{1}{Jet Propulsion Laboratory, California Institute of
  Technology, 4800 Oak Grove Dr, MS 169-506, Pasadena, CA 91109} 

\begin{abstract}
Methodology for inferring time delays in time-streams of photometric
measurements in a multiply imaged quasar.  Bayesian
inference. Comparison to other techniques including matrix inversion.
Realistic simulated light curves produced, and recovery efficiency
systematically explored, marginalizing over the nuisance parameters of
the parameters that control the behavior of the light curves.
Microlensing effects and considerations.  Main conclusions.
\end{abstract}
 
\keywords{gravitational lensing --- cosmology: dark matter} 

\section{Introduction}

% \begin{equation}
% {\rm S} = {{D_{\rm ang}(z_{\rm cluster}, z_{\rm target})}\over{D_{\rm ang}(z_{\rm target})}} 
%               {{D_{\rm ang}(z_{\rm model})}\over{D_{\rm ang}(z_{\rm cluster},z_{\rm model})}}. 
% \end{equation}

% \begin{figure*}[t]
% \begin{center}
% \plotone {figures/magfig_macs1149_11495_z9p7LAST.png}
% \caption{MACS 1149. Top left: The \emph{HST} ACS and WFC3 mosaic
%   footprint (inner and outer polygons), used to calculate the area
%   accessed by each. The image is the segmentation map. Remaining three
%   panels: The magnification map for source at the redshifts indicated,
%   with magnification strength shown in the bar on the right.
% }\label{fig:macs1149}
% \end{center}
% \end{figure*}

% \begin{figure}[t]
% \begin{center}
% \plotone{figures/mastervolume.png}
% %\includegraphics{../figures/b0712_model_for_masterlens.png}
% \caption{The cumulative volume accessible as a function of limiting
%   magnification, for a set of twelve CLASH clusters (and their sum). 
% }\label{fig:volume}
% \end{center}
% \end{figure}


\section{The Analysis Framework}\label{}
\subsection{Bayesian Inference}\label{}
\subsection{MCMC Sampling}\label{}

\section{Simulating Light Curves}\label{}
\subsection{Quasar physics}\label{}
\subsection{Observing cadences and campaigns}\label{}
\subsection{Photometric uncertainties}\label{}

\section{Running the Experiment}\label{}


\section{Discussions}\label{sec:disc}

\section{Conclusions}\label{sec:conc}

\acknowledgements

This work was carried out at Jet Propulsion Laboratory, California
Institute of Technology, under a contract with NASA.

\bibliographystyle{apj}
\bibliography{/Users/leonidas/Dropbox/bibdesk/moustakasbibs}

\end{document}

